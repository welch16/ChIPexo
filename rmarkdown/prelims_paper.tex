\documentclass[11pt]{article}\usepackage[]{graphicx}\usepackage[]{color}
%% maxwidth is the original width if it is less than linewidth
%% otherwise use linewidth (to make sure the graphics do not exceed the margin)
\makeatletter
\def\maxwidth{ %
  \ifdim\Gin@nat@width>\linewidth
    \linewidth
  \else
    \Gin@nat@width
  \fi
}
\makeatother

\definecolor{fgcolor}{rgb}{0.345, 0.345, 0.345}
\newcommand{\hlnum}[1]{\textcolor[rgb]{0.686,0.059,0.569}{#1}}%
\newcommand{\hlstr}[1]{\textcolor[rgb]{0.192,0.494,0.8}{#1}}%
\newcommand{\hlcom}[1]{\textcolor[rgb]{0.678,0.584,0.686}{\textit{#1}}}%
\newcommand{\hlopt}[1]{\textcolor[rgb]{0,0,0}{#1}}%
\newcommand{\hlstd}[1]{\textcolor[rgb]{0.345,0.345,0.345}{#1}}%
\newcommand{\hlkwa}[1]{\textcolor[rgb]{0.161,0.373,0.58}{\textbf{#1}}}%
\newcommand{\hlkwb}[1]{\textcolor[rgb]{0.69,0.353,0.396}{#1}}%
\newcommand{\hlkwc}[1]{\textcolor[rgb]{0.333,0.667,0.333}{#1}}%
\newcommand{\hlkwd}[1]{\textcolor[rgb]{0.737,0.353,0.396}{\textbf{#1}}}%

\usepackage{framed}
\makeatletter
\newenvironment{kframe}{%
 \def\at@end@of@kframe{}%
 \ifinner\ifhmode%
  \def\at@end@of@kframe{\end{minipage}}%
  \begin{minipage}{\columnwidth}%
 \fi\fi%
 \def\FrameCommand##1{\hskip\@totalleftmargin \hskip-\fboxsep
 \colorbox{shadecolor}{##1}\hskip-\fboxsep
     % There is no \\@totalrightmargin, so:
     \hskip-\linewidth \hskip-\@totalleftmargin \hskip\columnwidth}%
 \MakeFramed {\advance\hsize-\width
   \@totalleftmargin\z@ \linewidth\hsize
   \@setminipage}}%
 {\par\unskip\endMakeFramed%
 \at@end@of@kframe}
\makeatother

\definecolor{shadecolor}{rgb}{.97, .97, .97}
\definecolor{messagecolor}{rgb}{0, 0, 0}
\definecolor{warningcolor}{rgb}{1, 0, 1}
\definecolor{errorcolor}{rgb}{1, 0, 0}
\newenvironment{knitrout}{}{} % an empty environment to be redefined in TeX

\usepackage{alltt}
%\documentclass[11pt]{scrartcl}
\usepackage{amsmath, amsfonts, amssymb}
\usepackage{adjustbox}
\usepackage{fullpage}
\usepackage{epsfig}
%\setkomafont{disposition}{\normalfont\bfseries}
% \usepackage[round, sort]{natbib}
\renewcommand{\baselinestretch}{1.4}
\usepackage{float}


\title{High Resolution Identification of Protein-DNA Binding Events
  and Quality Control for ChIP-exo data\vspace*{\fill}}

\author{Rene Welch\\Preliminary Examination\\Department of Statistics, University of Wisconsin-Madison}

\date{December 1st, 2015}

%\usepackage{Sweave}
\IfFileExists{upquote.sty}{\usepackage{upquote}}{}
\begin{document}

\maketitle

\vspace*{\fill}

\textbf{Committee Members:}

\textbf{Professor S\"und\"uz Kele\c{s}}, Department of Statistics,
Department of Biostatistics and Medical Informatics

\textbf{Professor Karl Broman}, Department of Biostatistics and
Medical Informatics

\textbf{Professor Colin Dewey}, Department of Computer Sciences,
Department of Biostatistics and Medical Informatics

\textbf{Professor Christina Kendziorski}, Department of Biostatistics
and Medical Informatics

\textbf{Professor Ming Yuan}, Department of Statistics

\thispagestyle{empty}


\newpage

\tableofcontents

\newpage

\section*{Abstract}

    Recently, ChIP-exo has been developed to investigate protein-DNA
    interaction in higher resolution compared to popularly used
    ChIP-Seq. Although ChIP-exo has drawn much attention and is
    considered as powerful assay, currently, no systematic studies
    have yet been conducted to determine optimal strategies for
    experimental design and analysis of ChIP-exo. In order to address
    these questions, we evaluated diverse aspects of ChIP-exo and
    found the following characteristics of ChIP-exo data. First, the
    background of ChIP-exo data is quite different from that of
    ChIP-Seq data. However, sequence biases inherently present in
    ChIP-Seq data still exist in ChIP-exo data. Second, in ChIP-exo
    data, reads are located around binding sites much more tightly and
    hence, it has potential for high resolution identification of
    protein-DNA interaction sites, and also the space to allocate the
    reads is greatly reduced. Third, although often assumed in the
    ChIP-exo data analysis methods, the ``peak pair'' assumption does
    not hold well in real ChIP-exo data. Fourth, spatial resolution of
    ChIP-exo is comparable to that of PET ChIP-Seq and both of them
    are significantly better than resolution of SET ChIP-Seq. Finally,
    for given fixed sequencing depth, ChIP-exo provides higher
    sensitivity, specificity and spatial resolution than PET
    ChIP-Seq.

    In this article we provide a quality control pipeline which
    visually asses ChIP-exo biases and calculates a signal-to-noise
    measure. Also, we updated dPeak \cite{dpeak}, which makes a
    striking balance in sensitivity, specificity and spatial
    resolution for ChIP-exo data analysis.


\section{Introduction}
\label{sec:intro}

\section{Statistical Framework for ChIP-exo}
\label{sec:stat}



\section{Results}
\label{sec:results}



\section{Planned work}
\label{sec:future}





\newpage

\bibliographystyle{plain} % Style BST file (bmc-mathphys, vancouver, spbasic).
\bibliography{chip_exo_paper}

\end{document}
