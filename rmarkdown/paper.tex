% \VignetteEngine{knitr::knitr}
\documentclass{bmcart}\usepackage[]{graphicx}\usepackage[]{color}
%% maxwidth is the original width if it is less than linewidth
%% otherwise use linewidth (to make sure the graphics do not exceed the margin)
\makeatletter
\def\maxwidth{ %
  \ifdim\Gin@nat@width>\linewidth
    \linewidth
  \else
    \Gin@nat@width
  \fi
}
\makeatother

\definecolor{fgcolor}{rgb}{0.345, 0.345, 0.345}
\newcommand{\hlnum}[1]{\textcolor[rgb]{0.686,0.059,0.569}{#1}}%
\newcommand{\hlstr}[1]{\textcolor[rgb]{0.192,0.494,0.8}{#1}}%
\newcommand{\hlcom}[1]{\textcolor[rgb]{0.678,0.584,0.686}{\textit{#1}}}%
\newcommand{\hlopt}[1]{\textcolor[rgb]{0,0,0}{#1}}%
\newcommand{\hlstd}[1]{\textcolor[rgb]{0.345,0.345,0.345}{#1}}%
\newcommand{\hlkwa}[1]{\textcolor[rgb]{0.161,0.373,0.58}{\textbf{#1}}}%
\newcommand{\hlkwb}[1]{\textcolor[rgb]{0.69,0.353,0.396}{#1}}%
\newcommand{\hlkwc}[1]{\textcolor[rgb]{0.333,0.667,0.333}{#1}}%
\newcommand{\hlkwd}[1]{\textcolor[rgb]{0.737,0.353,0.396}{\textbf{#1}}}%

\usepackage{framed}
\makeatletter
\newenvironment{kframe}{%
 \def\at@end@of@kframe{}%
 \ifinner\ifhmode%
  \def\at@end@of@kframe{\end{minipage}}%
  \begin{minipage}{\columnwidth}%
 \fi\fi%
 \def\FrameCommand##1{\hskip\@totalleftmargin \hskip-\fboxsep
 \colorbox{shadecolor}{##1}\hskip-\fboxsep
     % There is no \\@totalrightmargin, so:
     \hskip-\linewidth \hskip-\@totalleftmargin \hskip\columnwidth}%
 \MakeFramed {\advance\hsize-\width
   \@totalleftmargin\z@ \linewidth\hsize
   \@setminipage}}%
 {\par\unskip\endMakeFramed%
 \at@end@of@kframe}
\makeatother

\definecolor{shadecolor}{rgb}{.97, .97, .97}
\definecolor{messagecolor}{rgb}{0, 0, 0}
\definecolor{warningcolor}{rgb}{1, 0, 1}
\definecolor{errorcolor}{rgb}{1, 0, 0}
\newenvironment{knitrout}{}{} % an empty environment to be redefined in TeX

\usepackage{alltt}

\usepackage{xcolor}
\usepackage{url}
\usepackage{amsmath}
\usepackage{amsthm}
\usepackage{amssymb}
\usepackage{graphicx}
\usepackage{tikz}
\usetikzlibrary{shapes,arrows}
\usepackage{float}
\usepackage{verbatim}
\usepackage{hyperref}
\hypersetup{
     colorlinks   = true,
     citecolor    = black
}
\hypersetup{linkcolor=black}
% \hypersetup{linkcolor=black}
% \hypersetup{linkcolor=black}



% \usepackage{authblk}
\usepackage[utf8]{inputenc} %unicode support

\newcommand{\sig}{\sigma^{70}}
\IfFileExists{upquote.sty}{\usepackage{upquote}}{}
\begin{document}

%\maketitle


\begin{frontmatter}

\begin{fmbox}
\dochead{Draft}

\title{High Resolution Identification of Protein-DNA Binding
  Events and Quality Control for ChIP-exo data}

\author[
   addressref={aff6},                   % id's of addresses, e.g. {aff1,aff2}
   email={chungd@musc.edu}   % email address
]{\inits{DC}\fnm{Dongjun} \snm{Chung}}
\author[
   addressref={aff1},
   email={welch@stat.wisc.edu}
]{\inits{RW}\fnm{Rene} \snm{Welch}}
\author[
   addressref={aff3},
   email={ong@cs.wisc.edu}
]{\inits{IO}\fnm{Irene} \snm{Ong}}
\author[
   addressref={aff3,aff4},
   email={jagrass@wisc.edu}
]{\inits{JG}\fnm{Jeffrey} \snm{Grass}}
\author[
   addressref={aff3,aff4,aff5},
   email={landick@bact.wisc.edu}
]{\inits{RL}\fnm{Robert} \snm{Landick}}
\author[
   addressref={aff1,aff2},
   corref={aff1},
   email={keles@stat.wisc.edu}
]{\inits{SK}\fnm{S\"und\"uz} \snm{Kele\c{s}}}



\address[id=aff1]{%                           % unique id
  \orgname{Department of Statistics, University of Wisconsin Madison}, % university, etc
  \street{1300 University Avenue},                     %
  %\postcode{}                                % post or zip code
  \city{Madison},                              % city
  \cny{WI}                                    % country
}

\address[id=aff2]{%
  \orgname{Department of Biostatistics and Medical Informatics, University of Wisconsin Madison},
  \street{600 Highland Avenue},
%  \postcode{24105}
  \city{Madison},
  \cny{WI}
}
\address[id=aff3]{
  \orgname{Great Lakes Bioenergy Research Center, University of Wisconsin Madison},
  \street{1552 University Avenue},
  \city{Madison},
  \cny{WI}
}
\address[id=aff4]{
  \orgname{Department of Biochemistry, University of Wisconsin Madison},
  \street{433 Babcock Drive},
  \city{Madison},
  \cny{WI}
}
\address[id=aff5]{
  \orgname{Department of Bacteriology, University of Wisconsin Madison},
  \street{1550 Linden Drive},
  \city{Madison},
  \cny{WI}
}
\address[id=aff6]{
  \orgname{Department of Public Health Sciences, Medical University of South Carolina},
  \street{135 Cannon Street},
  \city{Charleston},
  \cny{SC}
}


% \begin{artnotes}
% %\note{Sample of title note}     % note to the article
% \note[id=n1]{Equal contributor} % note, connected to author
% \end{artnotes}


\end{fmbox}

\begin{abstractbox}

  \begin{abstract}

  ChIP-exo is a modification of the ChIP-Seq protocol for high
  resolution mapping of transcription factor binding sites. Although
  many aspect of the ChIP-exo data analysis are similar to those of
  ChIP-Seq, ChIP-exo presents a number of unique challenges. We
  present a quality control pipeline that analyzes ChIP-exo's strand
  imbalance and library complexity. Assessment of these biases and
  artifacts are facilitated through diagnostic plots and summary
  statistics calculated over regions of the genome with varying levels
  of coverage.

  We systematically evaluated diverse aspects of ChIP-exo and found
  the following characteristics: First, ChIP-exo's background is quite
  different from ChIP-Seq's. Second, although often assumed in
  ChIP-exo data analysis methods, the ``peak pair'' assumptions does
  not hold locally in actual ChIP-exo data. Third, we for the first
  time compared Paired End (PE) ChIP-Seq with ChIP-exo and found that
  both protocols are comparable in resolutions and sensitivity for
  closely located binding events, but as the distance between binding
  events increases ChIP-exo shows higher sensitivity that PE
  ChIP-Seq. Finally, at fixed sequencing depths, ChIP-exo provides
  higher sensitivity, specificity and spatial resolution than PE
  ChIP-Seq.

  \end{abstract}

\begin{keyword}
  \kwd{ChIP-exo}
  \kwd{QC}
  \kwd{TFBS}
  \kwd{BS identification on High-Res}
\end{keyword}

\end{abstractbox}

\end{frontmatter}

\newpage

\listoffigures

\newpage

\listoftables

\newpage

\section{Background}
\label{sec:intro}

ChIP-exo (Chromatin Immunoprecipitation followed by exonuclease
digestion and next generation sequencing) Rhee and Pugh, 2011
\cite{exo1} is the state-of-the-art experiment developed to attain
single base-pair resolution of protein binding site identification and
it is considered as a powerful alternative to popularly used ChIP-Seq
(Chromatin Immunoprecipitation coupled with next generation
sequencing) assay. ChIP-exo experiments first capture millions of DNA
fragments (150 -250 bp in length) that the protein under study
interacts with using random fragmentation of DNA and a
protein-specific antibody. Then, exonuclease is introduced to trim the
5' end of each DNA fragment to a fixed distance from the bound protein
compared to ChIP-Seq. This step is unique to ChIP-exo and could
potentially provide significantly higher spatial resolution compared
to ChIP-Seq. Finally, high throughput sequencing of a small region (25
to 100 bp) at the 5' end of each fragment generates millions of reads.

While the number of ChIP-exo data keeps increasing,characteristics of
ChIP-exo data are not fully investigated yet. First, DNA libraries
generated by the ChIP-exo protocol seem to be less complex than the
libraries generated by ChIP-Seq (Mahony et al., 2015,
\cite{exo_review}), i.e. the possible number of positions to which the
reads can be aligned has been reduced due to the exonuclease
digestion. Second, although there are roughly the same amount of read
in both strands, locally there may be more reads in one strand than in
the other. To address this challenges, we suggest a collection of
quality control visualizations to understand which of this biases are
present in a ChIP-exo experiment and globally asses the enrichment and
library complexity of a ChIP-exo sample. We gathered ChIP-exo data from
diverse organisms: CTCF factor in human \cite{exo1}; ER factor in
human and FoxA1 factor in mouse (Serandour et al., 2013
\cite{exoillumina}); and generated $\sig$ factor in Escherichia Coli
(\emph{E. Coli}) under aerobic ($ + O_2$) condition, and treated by
rifampicin by 0 and 20 minutes (courtesy of Professor Robert Landick's
lab). %% note to self, we may add more experiments here

Most of current ChIP-Seq quality control (QC) guidelines (Landt et
al., \cite{encode_qc}) may not be applicable on ChIP-exo, additionally
to our knowledge there are not established QC pipelines for
ChIP-exo. Previous ChIP-exo analyses used ChIP-Seq samples to compare
the resolution between experiments (\cite{exo1}, \cite{exo2},
\cite{exoillumina}); Carroll et al., 2014 \cite{carroll.qc} studied
the use of the Strand-Cross Correlation (SCC) (Kharchenko et al., 2008
\cite{strandcc}) and showed that by filtering blacklisted regions the
estimation of the SCC is improved. However, using the SCC may not be
helpful since the peaks that are attained at the read and fragment
lengths are confused in a typical ChIP-exo SCC curve. Additionally
this method requires to know blacklisted regions in advance which may
not be available. In our pipeline we propose an out-the-shelf analysis
that contrasts the enrichment of the experiment against its library
complexity.

In order to obtain the potential benefits of ChIP-exo on protein
binding site identification, it is critical to understand which are
the important characteristics of ChIP-exo data and to use algorithms
that could fully utilize information available in ChIP-exo data. Rhee
and Pugh, 2011 \cite{exo1} discussed that reads in the forward and
reverse strand might construct peak pairs around bound proteins, of
which heights were implicitly assumed to be symmetric. Hence, they
used the ``peak pair method'' that predicts the midpoint of two modes
of peak pairs as potential binding sites. Mace (Wang et al, 20114
\cite{mace}), CexoR (Madrigal, 2015 \cite{cexor}) and peakzilla
(Bardet et al., 2013 \cite{peakzilla}), recently developed ChIP-exo
data analysis methods are also based on this peak pair
assumption. However, appropriateness of such assumption was not fully
evaluated in the literature yet. Furthermore, it is still unknown
which factors could affect protein binding site identification using
ChIP-exo data. In order to address this problem, we investigated
various aspects of ChIP-exo data by contrasting them with their
respective ChIP-Seq experiments.

Currently, research on statistical methods for ChIP-exo data is still
in its very early stage. Although many methods have been proposed to
identify protein binding sites from ChIP-Seq data (reviewed by
Wilbanks and Facciotti, 2012 \cite{evaluation} and Pepke and Wold,
2009 \cite{computation}), such as MACS (Zhang et al., 2008
\cite{macs}), CisGenome (Ji et al., 2008 \cite{cisgenome}) and MOSAiCS
(Kuan et al., 2009 \cite{mosaics}), these approaches reveal protein
binding sites in lower resolution, i.e., at an interval of hundreds to
thousands of base pairs. Furthermore, they report only one ``mode'' or
``predicted binding location'' per peak. Hence, these methods are not
appropriate to evaluate the potential of ChIP-exo data for high
resolution identification of protein binding sites. More recently,
deconvolution algorithms such as Deconvolution (Lun et al., 2009
\cite{csdeconv}), GEM (Guo et al., 2012 \cite{gem}, an improved
version of Guo et al., 2010 \cite{gps} ) and PICS (Zhang et al., 2010
\cite{pics}) have been proposed to identify binding sites in higher
resolution using ChIP-Seq data. However, most of them are still not
tailored for ChIP-exo and PE and SE ChIP-Seq data in a unified
framework and as a result, currently available methods are not
appropriate for fair comparison between ChIP-exo and ChIP-Seq. To
address these limitations, we developed an improved of dPeak (Chung et
al., 2013 \cite{dpeak}), a high resolution binding site identification
(deconvolution) algorithm that we previously developed for PE and SE
ChIP-Seq data, so that it can also handle ChIP-exo data. The dPeak
algorithm implements a probabilistic model that accurately describes
the ChIP-exo and ChIP-Seq data generation process.

In this work we demonstrate that the ``peak pair'' assumption of Rhee
and Pugh, 2013 \cite{exo2} does not hold well in real ChIP-exo
data. Furthermore, we found that when we analyze ChIP-exo data from
eukaryotic genomes, it is important to consider sequence biases
inherent to ChIP-exo data, such as mappability and GC content in order
to improve sensitivity and specificity of binding site
identification. We evaluated several methods to identify binding events
and dPeak performs competitively respect to GEM and MACE when
analyzing ChIP-exo data. More importantly, when comparable number of
reads is used for both ChIP-exo and ChIP-Seq , dPeak coupled with
ChIP-exo data provides resolution comparable to PE ChIP-Seq and both
significantly improve the resolution of protein binding identification
compared to SE-based analysis with any of the available methods.

\section{Results and discussion}
\label{sec:results}

\subsection{Deeply sequence E. Coli $\sig$ ChIP-exo and ChIP-Seq data}
 
$\sig$ factor is a transcription initiation factor of housekeeping
genesin E. Coli. In this organism's genomes, many promoters contain
multiple transcription start sites (TSS) and these TSS are often
closely spaced (10 $\sim$ 150 bp). These closely spaced binding sites
are considered to be multiple ``switches'' that differentially
regulate gene expression under diverse growth conditions
\cite{regulondb}. Therefore, investigation of ChIP-exo's potential for
identification and differentiation of closely spaced binding sites are
invaluable for elucidating the transcriptional networks of prokaryotic
genomes.

\subsection{Current ChIP-Seq guidelines and quality metrics on ChIP-exo data}

We calculated the most commonly used quality indicators (list in table
\ref{tab:qc}) for ChIP-Seq data using all the ChIP-exo data we
gathered. The PCR Bottleneck Coefficient (PBC) and the Normalized
Strand Cross-Correlation (NSC) are calculated as in \cite{encode_qc}.
Additionally, we calculated the Forward Strand Ratio (FSR) to comped
the number of fragments sequenced from each strand. We can observe
that the PBC is low, hence it would be incorrectly suggested to repeat
the experiment.

\begin{table}[H]
  \centering
  \caption{Usual quality control indicators applied to the gathered ChIP-exo samples. 
    PBC stands for PCR Bottleneck Coefficient, NSC for Normalized Strand Cross-Correlation and
    FSR for Forward Strand Ratio. We ommited the Relative Strand Cross-Correlation (RSC) because
    a typical ChIP-exo experiment is not accompanied by an input file.}
  \label{tab:qc}
\end{table}


\bibliographystyle{bmc-mathphys} % Style BST file (bmc-mathphys, vancouver, spbasic).
\bibliography{chip_exo_paper}

% \nocite{exo_gb}
%  \nocite{maplot1}
% \nocite{maplot2}
% \nocite{chipbeyond}
% \nocite{esl}



\end{document}

