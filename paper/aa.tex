Chromatin Immunoprecipitation followed by exonuclease digestion and
next generation sequencing (ChIP-exo) is currently one of the
state-of-the-art high throughput assays for profiling protein-DNA
interactions at or close to single base-pair resolution
\cite{exo1}. It presents a powerful alternative to popularly used
ChIP-seq (Chromatin immunoprecipitation coupled with next generation
sequencing) assay. ChIP-exo experiments first capture millions of DNA
fragments ($150$ - $250$ bps in length) that the protein under study
interacts with using a protein-specific antibody and random
fragmentation of DNA. Then, $\lambda$-exonuclease ($\lambda$-exo) is
deployed to trim the $5^{\prime}$ end of each DNA fragment to each
protein-DNA interaction boundary. This step is unique to ChIP-exo and
aims to achieve significantly higher spatial resolution compared to
ChIP-seq. Finally, high throughput sequencing of a small region ($36$
to $100$ bps) at the $5^{\prime}$ end of each fragment generates
millions of reads.  Similarly, ChIP-nexus (Chromatin
Immunoprecipitation followed by exonuclease digestion, unique barcode,
single ligation and next generation ligation) \cite{chipnexus} is a
further modification on the ChIP-exo protocol. ChIP-nexus aims to
overcome limitations of ChIP-exo by by yielding high complexity
libraries with numbers of cells comparable to that of ChIP-seq
experiments. This is achieved by reducing the numbers of ligations in
the standard ChIP-exo protocol from two to one, and adding unique,
randomized barcodes to adaptors to enable monitoring of
overamplification.
%where both sequencing adaptors are ligated at the
%end of the ChIP fragments. Then, after exonuclease digestion, DNA self-circularization with circLigase, and restriction enzyme cutting between the two adaptors, the final library is amplified. Compared to ChIP-exo, ChIP-nexus aims to   attain  higher resolution analysis while yielding higher complexity libraries.
Figure~\ref{fig:chip_diagram} illustrates the differences between
different ChIP-based protocols: ChIP-exo, single-end (SE) ChIP-seq,
paired-end (PE) ChIP-seq, ChIP-nexus. The $5^{\prime}$ ends of a
ChIP-exo/nexus experiment are clustered more tightly around the
binding sites of the protein than in a ChIP-seq experiment. In a PE
ChIP-seq experiment, both ends are sequenced as opposed to only the
$5^{\prime}$ end in a SE ChIP-seq. \SK{Note to Rene: ChIP-Nexus paper
  has a good description of what can go wrong with ChIP-exo; It seems
  like most of the read imbalance could also be due to ligation
  inefficiency. Although this probably does not explain why we have
  reads only from one strand in some regions. This is more likely
  explained by over digestion in one strand or single stranded TF-DNA
  interaction. Think a bit more on these issues and understand how the
  ligation could give rise to strand imbalance, we can ask Bob to have
  a closer look at the section where we have this discussion.}

Although ChIP-exo/nexus protocols are being adopted by research
community, features of ChIP-exo data, especially those pertaining to
data quality, have not been investigated much. The key features of
ChIP-exo/nexus that separate them from ChIP-seq are broadly as
follows. First, DNA libraries generated by the ChIP-exo protocol are
expected to be less complex than the libraries generated by ChIP-seq
\cite{exo_review}) because digestion by $\lambda$-exo is expected to
restrict the space of genomic positions that sequencing reads can map
to, to small local regions around the actual binding sites. Therefore,
in high quality and especially deeply sequenced ChIP-exo datasets, it
is possible to observe large numbers of reads accumulating at a small
number of bases due to actual signal rather than overamplification
bias as commonly observed in ChIP-seq experiments.  Second, although
we expect approximately the same numbers of reads from both DNA
strands at a given binding site, there may be locally more reads in
one strand than in the other, owing to $\lambda$-exo efficiency,
ligation efficiency, or other factors.  This is a key point with
implications on the statistical analysis of ChIP-exo
data. Specifically, currently available ChIP-exo specific statistical
analysis methods (e.g., Mace \cite{mace}, CexoR \cite{cexor}, and
Peakzilla \cite{peakzilla}) rely on the existence of peak-pairs formed
by forward and reverse strand reads at the binding site.  Finally,
most of current widely used ChIP-seq quality control (QC) guidelines
\cite{encode_qc} may not be directly applicable to ChIP-exo data.

To address these challenges, we develop a suite of diagnostic plots
and summary statistics and implement them in a versatile \texttt{R
  package} named \pname{}.  We apply this pipeline on a large
collection of public and newly generated ChIP-exo/nexus data. We
validate implications of the QC pipeline by evaluation of the samples
for features that capture high signal to noise such as occurrences of
motifs recognized by the profiled DNA interacting protein. Our
analyses of this large collection of data revealed that the so-called
ChIP-exo peak-pair assumption is subject to violations. To further
address this and provide a platform where ChIP-exo and ChIP-seq
experiments can be evaluated with comparable methods, we assess
performances of recently developed methods suitable for ChIP-exo
analysis, including dpeak \cite{dpeak} and GEM \cite{gem}. We observe
that dPeak performs as good or better than the available ChIP-exo
methods and provides a platform where PE and SE ChIP-seq can be
compared with their ChIP-exo counterpart. Our comparisons of PE
ChIP-seq with ChIP-exo interestingly highlights that while ChIP-exo
outperforms PE ChIP-seq in terms of resolution and detection power,
both are significantly better than SE ChIP-seq.
